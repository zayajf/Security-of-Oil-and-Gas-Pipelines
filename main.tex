\documentclass{article}
\usepackage[utf8]{inputenc}

\title{Security of Oil and Gas Pipelines}
\author{Group 14}
\date{April 25 2021}

\begin{document}


\maketitle{Abstract:} What are Pipelines? Pipelines are primarily instrumental for the transportation and distribution of bulk products from one part of a country too another,  as well as removing finished products from refineries to terminals and depots (Hasan 2016). Pipelines are especially useful for transporting petroleum products, which are high and rising in value. Because petroleum is viewed as such a high value commodity. These particular pipelines are targets for pilferers. Any attempt to pilferage or sabotage these pipelines can lead to serious implications in the distribution of petroleum and petroleum products can cause major risk to people and especially the environment. It is extremely important that these pipelines have the proper implications in place. In our survey we will be sharing the different protocols and procedures that are in place to assure the security  of Oil and Gas Pipelines. We will talk about the risk-based models, in line monitoring systems,............ (add your topics here )

\section{Introduction}
Cross-country pipelines are the most energy-efficient, safe, environmentally friendly, and economic way to ship hydrocarbons (gas, crude oil, and finished products) over long distances either within the geographical boundary of a country or beyond it (Hasan 2016). Studies shows a significant portion of many nations' energy requirements are now transported through pipelines. Pipelines represent critical infrastructures for a successful oil and gas industry. It is pivotal to to the success distribution of sustained energy supply and economic growth (Walker et al. 1995) The economy of a country is heavily dependent on smooth and uninterrupted operation of these lines (Dey and Gupta 2000). determine the effectiveness of the entire business. The network of pipelines indisputably out-rates other transportation modes such as truck/trains due to its cost effectiveness, convenience, high land-use efficiency, higher reliability, higher degree of safety and security, and environmental friendliness over great distance ( Hossam and Hossam 2010). Cross-country pipelines follow a dynamic environment en route to their path, through outskirts of cities, villages, marshy areas, rivers, uninhabited remote barren lands, and tough terrains, thereby making them vulnerable and easy targets for antisocial and criminal activities like pilferage, sabotage, and terrorist attacks. (Hasan 2016). 

\section{Risk-Based Model}

\section{In-Line Detection}

\end{document}
